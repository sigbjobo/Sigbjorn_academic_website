\documentclass[12pt]{article}
\usepackage{hyperref}
\usepackage[utf8]{inputenc}
\usepackage[T1]{fontenc}
\usepackage{graphicx}
\hypersetup{colorlinks=true,linkcolor=blue, linktocpage}
\begin{document}
\section*{Risengrynsgr{\o}t, 25.02.2019}
\begin{itemize}
\item Oppskrift:
  \begin{itemize}
  \item Brukte ingredienser i tr{\aa}d med
    \href{https://www.melk.no/Oppskrifter/Groeter/Tradisjonsgroet/Risengrynsgroet}{oppskrift},
    men kokte gr{\o}ten i vannbad ved hjelp av to gryter. 
  \end{itemize}
\item Kommentarer:
  \begin{itemize}
  \item Ganske forn{\o}yd med smaken, men mest av alt forn{\o}yd med hvor lite arbeid som krevdes i og med at det er umulig å svi gr{\o}ten.
  \item Tok kanskje litt over en time.
  \end{itemize}
\item Konklusjon:
  \begin{itemize}
  \item Smakte veldig godt, men har laget bedre f{\o}r. Jeg tror
    kanskje det er noe {\aa} g{\aa} p{\aa} i starten, hvor den kokes for
    hardt i vann. Neste gang vil jeg starte direkte i melk. Man kan
    alltids spekulere i hvorvidt  å koke gr{\o}ten i vannbad
    kanskje gjør den litt tammere. Ellers har jeg lyst {\aa} pr{\o}ve
    {\aa} vaske risen, selv om jeg frykter dette vil ta bort
    stivelse. En annen ting er risen - typen og holdbarhet. Denne
    gangen brukte jeg extrarisen (den beste s{\aa}langt.
  \end{itemize}
\end{itemize}

\end{document}

